%; whizzy-master slides.tex
%; whizzy frame -pdf ./skimopen 

\usepackage{xspace} % To get the right spacings in front of : and so on
\usepackage[english]{babel}
\usepackage{subfigure}
\usepackage{pgf}
\usepackage{colortbl}
\usepackage{xcolor}
\usepackage{abbrevs}
\usepackage[color]{coqdoc}
\def\coqlibrary#1#2{}

\usepackage{coq}
\usepackage{tikz}
\usepackage{pgfplots}
\usetikzlibrary{arrows,shapes,positioning,calc,scopes,
  fit,automata}
\usepackage{pifont}
\usepackage{utf}
\usepackage{natbib}
\usepackage[OT1]{fontenc}
\usepackage{alltt}
\def\thetitle{\Coq 8.5: what's new, what's next?}
\def\names{M. Dénès, M. Sozeau}
\def\theauthor{\name{M. Dénès$^{\dagger}$} \& \name{M. Sozeau$^{*}$}}
\usepackage{hyperref}


\def\vec#1{\overrightarrow{#1}}
\def\thetextitleabbr{\thetitle}
\def\thetextitle{\thetextitleabbr}
\def\thesubject{Theoretical Computer Science}
\def\thekeywords{Coq}
\def\theevent{\Coq POPL Meeting\\
January 21st 2014\\
San Diego, USA}
\def\theeventabbr{}
\def\theauthorabbr{\name{M. Dénès} \& \name{M. Sozeau}}
\usepackage{prelude-beamer}

\author{\theauthorabbr}
\def\PI{\name{PI}}
\def\shaded#1{{\color{black!50}#1}}
 
\def\seq{\fCenter}

\pgfdeclareimage[height=1.4cm]{inria-logo}{inria-logo}

\pgfdeclareimage[height=3cm]{coq-biglogo}{barron-logo-pdf}
\pgfdeclareimage[height=0.4cm]{coq-smalllogo}{barron-logo}
\pgfdeclareimage[height=1cm,interpolate=true]{coq-logo}{barron-logo-pdf}

\pgfdeclareimage[height=7cm]{coq-share}{coq-share}

\def\imgheight{7.5cm}

\def\typea{\vdash}
\def\suba{\rightslice}
\def\indent{\coqdocindent}
\def\var{\coqdocvar}

\def\coqdocid#1{\coqdocvar{#1}}

%\renewcommand{\Type}[1]{\kw{Type}_{#1}}

\newcommand{\elt}[4]{\ensuremath{\constr{exist}_{#1,#2}~#3~#4}}
\newcommand{\eltlight}[2]{\ensuremath{\constr{exist}~#1~#2}}
\def\class{\kw{class}~}
\def\instance{\kw{instance}~}
\def\tclass#1{\coqdocind{#1}}
\def\where{\kw{where}~}
\def\module#1{\texttt{#1}}

\def\vector#1{\ind{vector}~#1}
\def\vnil{\constr{vnil}}
\def\vcons{\constr{vcons}}

\renewcommand{\bar}[1]{{\overline{#1}}}

\setlength{\coqdocbaseindent}{0em}

\def\coqlibrary#1#2#3{}

% Tutorial

\newboolean{darkfond}
\setboolean{darkfond}{false}
%\setboolean{darkfond}{true}

\definecolor{metavarcolor}{rgb}{0.5,0.0,1.0}
\definecolor{darkgreen}{rgb}{0.1,0.7,0.1}
\definecolor{answercolor}{rgb}{1.0,0.0,0.0}
\definecolor{normalcolor}{rgb}{0.0,0.0,0.0}
\definecolor{exbluecolor}{rgb}{0.1,0.1,0.9}
\definecolor{dontlookcolor}{rgb}{0.5,0.5,0.5}
\definecolor{termcolor}{rgb}{0.0,0.1,0.9}
\definecolor{lookcolor}{rgb}{0.8,0.2,0.0}
\definecolor{prooftermcolor}{rgb}{0.3,0.1,1.0}
\definecolor{typecolor}{rgb}{1.0,0.6,0.0}
\definecolor{taccolor}{rgb}{0.70,0.10,0.0}
\definecolor{pink}{rgb}{0.8,0.6,0.6}
\definecolor{darkmagenta}{rgb}{0.4,0.0,0.6}
\definecolor{darkblue}{rgb}{0.0,0.0,0.6}

%\newenvironment{subsecframe}[2][]{\begin{frame}[environment=subsecframe,#1]\frametitle{#2}\begin{subsection}{#2}}{\end{subsection}\end{frame}}

\newenvironment{subsecframe}[1]{\begin{subsection}{#1}\begin{frame}\frametitle<presentation>{#1}}{\end{frame}\end{subsection}}

\newenvironment{subsecframefragile}[1]{\begin{frame}[fragile,environment=subsecframefragile]\frametitle<presentation>{#1}\begin{subsection}{#1}}{\end{subsection}\end{frame}}

\usepackage{xparse}

\DeclareDocumentCommand\demoframe{m m g g}{\only<presentation>{%
  \begin{frame}\begin{center}{\Huge #1}\\\bigskip{\Large #2}%
      \IfNoValueF{#3}{\\\vspace{2em}\includegraphics[height=10em]{#3}\IfNoValueF{#4}{\hspace{2em}\includegraphics[height=10em]{#4}}}\end{center}\end{frame}}}


%%%%%%%%%%%%%%%%%%%%%%%%%%%
\mode<presentation>

\setbeamertemplate{background canvas}[vertical shading][top=rouge1,bottom=rouge1,middle=rouge1]
\setbeamercolor{toto}{fg=blanc,bg=rouge1}

\setbeamertemplate{footline}
{
\begin{beamercolorbox}[wd=1\paperwidth,ht=15.5pt]{toto}
%\raisebox{1ex}{\includegraphics[width=25mm]{logobastrans}}
  \hspace{4em}
  \raisebox{2.5ex}
  {\thetitle{}  -- \theauthorabbr{}}\hfill 
  \raisebox{2.5ex}
  {\insertframenumber \hspace{2mm} \null }
\end{beamercolorbox}}

\setbeamertemplate{background canvas}[vertical shading][top=white,middle=white,bottom=white]

\def\bulletfail{\alert{\ding{54}}}
\def\bulletcheck{\ding{52}}

%\setbeamercolor{background}{fg=red,bg=white,text=blue}

%\setbeamercolor{subsection in toc shaded}{fg=gray,bg=white,text=blue}
\setbeamercolor{section in toc}{fg=black,bg=white}
\setbeamercolor{subsection in toc}{fg=black,bg=white}
%\setbeamertemplate{table of contents shaded}[default]
%\setbeamertemplate{subsection in toc shaded}{\textcolor{gray}}

\setbeamertemplate{section in toc shaded}
{\begin{colormixin}{20!white}{\usebeamertemplate{section in toc}}\end{colormixin}\unskip}

\setbeamertemplate{subsection in toc shaded}
{\begin{colormixin}{20!white}{\usebeamertemplate{subsection in toc}}\end{colormixin}\unskip}

\setbeamertemplate{subsubsection in toc shaded}
{\begin{colormixin}{20!white}{\usebeamertemplate{subsubsection in toc}}\end{colormixin}\unskip}

\setbeamertemplate{background canvas}[vertical shading][top=rouge1,middle=rouge1,bottom=rouge1]
\setbeamertemplate{footline}{\hspace{5em} \textcolor{white} {\null
    \hfill}\hspace{2em}\null \vspace*{3pt}}


% Graphics
\pgfplotsset{
   tufte ybar/.style={
      ybar,
%      hide x axis,
      axis line style={opacity=0},
      major tick style={draw=none},
      ymin=0,
      bar width=0.8em,
      ymajorgrids, tick align=inside,
      major grid style=white,
      axis on top,
      axis y line*=right,
%      after end axis/.code={
%        \draw [very thick] ({rel axis cs:0,0}-|{axis cs:\pgfplots@data@xmin,0})
%              ++(-0.5*\pgfkeysvalueof{/pgf/bar width},0pt)
%              -- ({rel axis cs:0,0}-|{axis cs:\pgfplots@data@xmax,0})
%              -- ++(0.5*\pgfkeysvalueof{/pgf/bar width},0pt);
%      },
%      yticklabel=\pgfmathprintnumber{\tick}\,\%,
      ytick={20,40,...,100},
      tickwidth=0pt,
      legend style={draw=none},
      legend style={
          at={(0.5,-0.1)},
          anchor=north,
          legend columns=-1,
          /tikz/every even column/.append style={column sep=0.5cm}
      },
      enlarge y limits={value=.1,upper},
      enlarge x limits=true,
      nodes near coords={
        \pgfmathprintnumber[precision=0]{\pgfplotspointmeta}
       }
   }
}


\mode<all>

\begin{document}




\mode
<presentation>
{\begin{frame}


\begin{textblock*}{40mm}[0,0](10mm,4mm)
 \includegraphics[width=6cm]{INRIA_SCIENTIFIQUE_UK_BLANC_fond-transparent}
  \end{textblock*}

\begin{textblock*}{12cm}(13mm,50mm)
{\textcolor{white} {
{\huge \thetitle}\\[2mm]
{\theauthor}}}
\end{textblock*}


   \begin{textblock*}{40mm}[0,0](10mm,76mm)
  \begin{picture}(5,80)
\put(0,23){\includegraphics[width=4cm,height=1.5cm]{logobasrougeV1}}
\put(20,50){\tiny \textcolor{rouge2}{${}^{\dagger}$University of Pennsylvania}}
\put(20,40){\tiny \textcolor{rouge2}{${}^{*}$$πr^2$ - Inria \& PPS}}
\end{picture}
\end{textblock*}


\begin{textblock*}{7cm}(55mm,76mm)
{\textcolor{white}{{\theevent}}}
\end{textblock*}

\vspace*{-4pt}
\end{frame}


\setbeamertemplate{footline}
{
\begin{beamercolorbox}[wd=1\paperwidth]{toto}
%\raisebox{1ex}{\includegraphics[width=25mm]{logobastrans}}
  \hspace{4em}
  \raisebox{2.5ex}
  {\thetitle{}  -- \theauthorabbr{}}\hfill 
  \raisebox{2.5ex}
  {\insertframenumber \hspace{2mm} \null }
\end{beamercolorbox}}

\setbeamertemplate{background canvas}[vertical shading][top=white,middle=white,bottom=white]

}

\frame<beamer>{\frametitle{\Coq 8.5}
\tableofcontents[sectionstyle=show/show,hideallsubsections]}

\section{Up and coming}

\demoframe{Incremental development}{Enrico Tassi}{tassi.jpg}

\begin{subsecframe}{Incremental development -- \name{E. Tassi}}
  \begin{center}
    Asynchronous and parallel processing of definitions.    
    Huge gain in user productivity.
  \end{center}

  \mode<article>
  \begin{itemize}
  \item \texttt{git} model
  \item Granularity: opaque lemmas
  \item \texttt{.vi} interfaces: cross-file compilation
  \item CoqIDE only (Emacs interface hard to implement)
  \end{itemize}
  \mode<all>

  \begin{itemize}
  \item[--/++] \emph{Not optional, backwards-compatible}
  \item[+++] \emph{Faster interaction and parallel compilation}
  \end{itemize}
\end{subsecframe}

\demoframe{New proof engine}{Arnaud Spiwack}{spiwack.jpg}

\begin{subsecframe}{New proof engine -- \name{A. Spiwack}}
  \begin{center}
    Clear proof-search semantics and dependent subgoals.
  \end{center}
  
  \mode<article>
  \begin{itemize}
  \item Existential variables + \texttt{LogicT}
  \item Proof-search semantics (\texttt{;}, \texttt{+}, $\ldots$)
  \item Cleaner code: abstract monadic interface
  \item Native \texttt{refine}, dependent goals, goal manipulation
  \end{itemize}
  \mode<all>

  \begin{itemize}
  \item[--/++] \emph{Not optional, backwards-compatibility layer}
  \item[--] \emph{0-15\% time overhead (still investigating, \textsc{OCaml} issue likely)}
  \end{itemize}
\end{subsecframe}

\demoframe{Universe polymorphism}{Matthieu Sozeau}{sozeau}

\begin{subsecframe}{Universe polymorphism -- \name{M. Sozeau}}
% Def interactive
  \begin{center}
    Truly polymorphic definitions and inductives, cleaner kernel. 
  \end{center}
  
  \mode<article>
  \begin{itemize}
  \item \emph{Checked} universes.
  \item Global definitions (and inductives) or local with prenex, ``bounded''
    quantification. \emph{Conservativity proof}
  \item \emph{Elaboration} from typical ambiguity + implicit
    generalization.
  \item Checks \texttt{HoTT/Coq}, \name{B. Barras} and \name{T. Coquand}'s semi-simplicial
    model and another (1-)groupoid model.
  \item Better error messaging, unification sensitive to universes.
  \item Generalized rewriting in \alert{\texttt{Type}}
  \end{itemize}
  \mode<all>

  \begin{itemize}
  \item[=/++] \emph{Kernel change - impacts the ML hacker only}. Backwards-compatibility layer.
  \item[=+/+] \emph{Comparable or better performance, more expressive}
  \end{itemize}
\end{subsecframe}

\demoframe{Native compilation}{Maxime Dénès \& Benjamin Grégoire}{denes}{gregoire}

\begin{subsecframefragile}{Native compilation -- \name{M.~Dénès \&
      B.~Grégoire}}
\begin{tikzpicture}
\begin{axis}
  [
  tufte ybar,
  symbolic x coords={FourColor,Lucas,RecNoAlloc,Cooper},
  ylabel={Relative time (\%)},
  height=6cm, width=10cm,
  xtick=data,
  legend style={fill=none}
]
\addplot[draw=none,fill=black] table[x index=0,y expr=\thisrowno{1}/\thisrowno{1}*100] {data/native.dat};
\addplot[draw=none,fill=gray] table[x index=0,y expr=\thisrowno{2}/\thisrowno{1}*100] {data/native.dat};
\addplot[draw=none,fill=gray!50!white] table[x index=0,y expr=\thisrowno{3}/\thisrowno{1}*100] {data/native.dat};

  \legend{VM,Native,Extracted}

\end{axis}
\end{tikzpicture}
\end{subsecframefragile}

\begin{frame}[fragile]
  \frametitle{Native compilation -- \name{M.~Dénès \&
      B.~Grégoire}}
  \begin{center}
    Down to assembly through \name{OCaml}. Useful for large reflection proofs.
  \end{center}

  \mode<article>
  \begin{itemize}
  \item \Coq $"->"^{\texttt{nativecomp}}$ \textsc{ML} $"->"^{\texttt{ocamlopt}}$ \textsc{ASM}
  \item Faster than \texttt{vm\_compute}, to be merged.
  \end{itemize}
  \mode<all>

  \begin{itemize}
  \item[+] \emph{Optional}
  \item[++/--] \emph{Faster at runtime, compilation is slow}
  \end{itemize}
\end{frame}

\demoframe{Fast record projections}{Matthieu Sozeau}

\begin{subsecframefragile}{Fast record projections -- \name{M. Sozeau}}

  \begin{center}
    Faster conversion and type-checking, smaller memory
    footprint.
  \end{center}
  
  \mode<article>
  \begin{itemize}
  \item Economic representation
  \item η-rule (for non-empty records only): 
    \verb|Build_rec (p1 r) .. (pn r)| is canonical
  \item Fast reduction (address of \verb|r| + offset)
  \item<article> No more stupid unfoldings
  \end{itemize}
  \mode<all>
    
  \begin{itemize}
  \item[+/+--] \emph{Optional, backwards-compatibility layer,
      small source-level incompatibilities}.
  \item[${+}^{ω}$] \emph{Exponentially better performance}
  \end{itemize}

\end{subsecframefragile}

\demoframe{New coqdoc}{Yann Régis-Gianas}{regis-gianas}
\begin{subsecframefragile}{New coqdoc -- \name{Y. Régis-Gianas}}
\frametitle{A new architecture}
\begin{center}
\includegraphics[width=10cm]{coqdoc.png}
\end{center}
\end{subsecframefragile}

\begin{frame}[fragile]
\frametitle{New coqdoc -- \name{Y. Régis-Gianas}}

  \begin{center}
    A generic documentation tool based on the existing \Coq parser.
  \end{center}
  \mode<article>
  \begin{itemize}
  \item The current implementation \texttt{coqdoc} has several flaws:
    \begin{itemize}
    \item It reimplements an approximate parser for .v files.
    \item It generates documents that are hard to customize.
    \item It contains many bugs.
    \end{itemize}
  \item Following \texttt{coqide}'s philosophy, this tool should
    only focus on the document generation and delegates the syntactic
    and semantic analysis to \texttt{coqtop}.
    
  \item By the way, similar remarks are valid for \texttt{coqtex} too.
  \end{itemize}

  Planning:
  \begin{itemize}
  \item A prototype has been implemented by Fran\c{c}ois Ripault during an
    internship last year. A ready-for-integration implementation will
    be pushed in the trunk by the end of february.
  \item In the next release, we will ship a \verb!coqdoc! embedding
    the old and the new implementation with a flag to choose between
    them.
  \end{itemize}
  \mode<all>

  \begin{itemize}
  \item[+] \emph{Not optional but legacy \texttt{coqdoc} code included}
  \item[=] \emph{Same performance}
  \end{itemize}
  
  % \vfill
  % \begin{center}\alert{{\Huge Demo}}\end{center}
\end{frame}

\begin{subsecframefragile}{Misc -- \Coq dev team \& contributors}
  \begin{itemize}
  \item Interfaces, documentation, and \textsc{OCaml} best practices
    (\name{P.M. Pédrot}, $\ldots$). \alert{Performance is better than 8.4}
  \item Tactics in terms: \verb|$(tac)$| (\name{P.M. Pédrot})
  \item Module system simplifications (\name{P. Letouzey})
  \item Tactic improvements (e.g. intro patterns in \texttt{injection})
    (\name{H. Herbelin, P. Letouzey, \dots})
  \item More expressive guard condition (\name{P. Boutillier, H. Herbelin})
  \item Rewriting with strategies (\name{M. Sozeau})
  \end{itemize}
  
  \begin{center}
    \includegraphics[height=5em]{pedrot}
    \hspace{2em}\includegraphics[height=5em]{letouzey}
    \hspace{2em}\includegraphics[height=5em]{herbelin-ias}
    \hspace{2em}\includegraphics[height=5em]{boutillier.png}
  \end{center}
\end{subsecframefragile}

\demoframe{\texttt{opam}}{Thomas Braibant}{braibant}
\begin{subsecframe}{\texttt{opam} -- \name{T. Braibant}}
  
  \begin{itemize}
  \item \name{Coq, HoTT/Coq, Coq+Ssreflect}, any version, \texttt{git} included
  \item Some user contribs: \name{Containers, Coccinelle, Ergo, \dots}
  \item Submit a pull request! Try! Test!
  \end{itemize}

  \begin{center}
    \url{http://opam.ocaml.org/}
    \vspace{0.5em}

    \url{https://github.com/braibant/opam-coq-repo}
  \end{center}
\end{subsecframe}

\begin{subsecframe}{Tentative planning}
  \begin{itemize}
  \item Late February 2014: 8.5-α release and 8.4pl4 (or 8.4-2)
  \item April 2014: 8.5-β release and opening of new branch
    (only bug fixes in trunk between α and β)
  \item August 2014: final release
  \end{itemize}
\end{subsecframe}


\section{The close future}
\frame<beamer>{\frametitle{The close future}
\tableofcontents[sectionstyle=show/shaded,hideallsubsections]}

\begin{subsecframe}{Performance}

  \begin{itemize}
  \item A sliding scale for efficiency/safety in the kernel (ideas:
    tainting code/data, signatures). Allowing, unchecked recursive
    definitions, custom reductions\dots
  \item Machine integers and persistent arrays (\name{M. Dénès, B.~Grégoire})
  \item Caching of typechecked terms (\name{E. Tassi, M. Sozeau}):
    typecheck once, reuse as many times at 0 cost! (\TODO{slides enrico
      GT conversion congruence}).
  \item Maximal sharing / generalized hash-consing of terms (\name{T. Braibant, M. Sozeau,
      P. M. Pédrot, $\ldots$})
  \end{itemize}
  
\end{subsecframe}

\begin{subsecframe}{CoqMT -- \name{P.Y. Strub} et al}
  \begin{itemize}
  \item Extend the definitional equality with a (first-order) decidable
    theory, e.g. Presburger arithmetic, constructors
  \item<presentation:only@0>[$"->"$] A \emph{strict} equality avoiding coercions.
    If used, a slightly larger TCB (SN proof available).
  \item Some limitations on the use of equalities in context
  \item Current state: useful but not as well-integrated as it could
    be, still at 8.3.
  \item To be integrated as an optional feature
  \end{itemize}

  \begin{itemize}
  \item[+] \emph{Optional}
  \item[+/--] \emph{Smoother programming with dependent types, slows
      down unification/conversion}
  \end{itemize}

\end{subsecframe}  

\section{The distant future}
\frame<beamer>{\frametitle{The distant future}
\tableofcontents[sectionstyle=show/shaded,hideallsubsections]}

\begin{subsecframe}{HoTT or OTT?}

  \begin{itemize}
  \item 
    HoTT ideas, math oriented (an ideal foundation):
    \begin{itemize}
    \item HITs (prototype by \name{B. Barras}, \name{J. Gross} will
      explain).
    \item<presentation:only@0> \texttt{hProp}/\texttt{hSet}, resizing rules
    \item<presentation:only@0> \alert{\emph{Definitionally}}: βδ, η?, ι? 
    \item \alert{\emph{Propositionally}}: proof-irrelevance, functional extensionality,
      isomorphic types are equal.
    \end{itemize}
  \item 
    OTT/Epigram ideas, CS oriented (another ideal foundation):
    \begin{itemize}
    \item Generic programming on inductive types universe.
    \item<presentation:only@0> Clear $\Prop$/$\Set$ separation
    \item \alert{\emph{Definitionally}}: βδηι, proof-irrelevance.
    \item \alert{\emph{Propositionally}}: \mode<article>{
      Propositional and functional extensionality.  defined, heterogeneous on values,
      structural on types ($\neq$ iso). With coercion and coherence operators.}
      \mode<presentation>{Functional extensionality, iso. types are \emph{not} equal}
    \end{itemize}
  \end{itemize}
  
  Ideally, we should mix the two.
\end{subsecframe}

\begin{subsecframe}{\Coq as a programming language}

  Usability, efficiency woes:
  \begin{itemize}
  \item I/O, FFI (with tainting)
  \item Proof-irrelevance in conversion: gives ``true'' dependent
    pattern-matching, true subset/refinement types.
  \item Type-based termination (sized-types, \name{J. L. Sacchini},
    \alert{postdoc position available} at CMU in Qatar!)
  \end{itemize}
\end{subsecframe}



% \begin{subsecframe}{Consortium}
% \end{subsecframe}
% \begin{subsecframe}{Development model}
%   Recentralize (\name{P. Letouzey}, \Coq dev team)
%   \begin{itemize}
%   \item , bugzilla, website on new machine (TODO by Pierre L), patches, dependencies of Coq, publishing libraries TODO : move to discussion ?


\begin{subsecframe}{User libraries, plugins}

  \begin{itemize}
  \item \name{Ssreflect} 1.5 (compatibility with TCs, split between
    \name{ssr} and \name{math-comp}, more theories: \url{http://www.msr-inria.fr/projects/mathematical-components/})
  \item \name{Mtac}, \name{Cybele}: ``internalized'' tactic language
    (demo by \name{B.~Ziliani})
  \item \name{Paco}: co-induction, logically
  \item \name{Extlib}, \name{TLC}: extensions of the standard library
    (demos by \name{G.~Malecha}, \name{A.~Charguéraud})
  \item \name{Why3}: call SMT solvers from \Coq (demo by \name{J.C.~Filliâtre})
  \item \name{CoqMT} demo by \name{P.Y.~Strub}
  \item \name{Equations}: dependent pattern-matching compilation (demo
    by \name{M.~Sozeau})
  \end{itemize}
  
\end{subsecframe}

\begin{subsecframe}{The End}
  \begin{center}
    \pgfuseimage{coq-biglogo}
  \end{center}
\end{subsecframe}

\begin{subsecframe}{The Bug}
  \begin{itemize}
  \item Due to a (very) extensional axiom, related to \emph{equality} on
    types interacting with the untyped guard condition.
  \item A fix is known, but breaks a few legitimate uses of the recently
    extended guard condition in the stdlib. All definitions can be
    adapted. Work by \name{B. Barras, C. Paulin, M. Dénès,
      P. Boutillier}, $\ldots$
  \item[=] Happens in \name{Agda} too!
  \item Argues for less syntax and more logic and/or
    certified metatheoretical proofs.
  \end{itemize}
\end{subsecframe}

\end{document}

%%% Local Variables: 
%%% mode: latex
%%% TeX-PDF-mode: t
%%% TeX-master: "slides"
%%% End: 
