%; whizzy-master slides.tex
%; whizzy frame -pdf ./skimopen 

\usepackage{xspace} % To get the right spacings in front of : and so on
\usepackage[english]{babel}
\usepackage{subfigure}
\usepackage{bussproofs}
\usepackage{pgf}
\usepackage{colortbl}
\usepackage{xcolor}
\usepackage{abbrevs}
\usepackage[color]{coqdoc}
\def\coqlibrary#1#2{}

\usepackage{hyperref}
\usepackage{coq}
\usepackage{code}
\usepackage{codecolor}
\usepackage{cond}
\usepackage{me}
\usepackage{tikz}
\usetikzlibrary{shapes}
\usepackage{pifont}
\usepackage{proof}
\usepackage{utf}
\usepackage{natbib}
\usepackage[OT1]{fontenc}
\usepackage{alltt}
\usepackage{amsthm}


\input{coqmacros}
%\input{univmacros}

\def\vec#1{\overrightarrow{#1}}
\def\thetitle{\Coq 8.5: what's new, what's next?}
\def\thetextitleabbr{\thetitle}
\def\thetextitle{\thetextitleabbr}
\def\thesubject{Theoretical Computer Science}
\def\thekeywords{Coq}
\def\theevent{\Coq POPL Meeting\\
January 21st 2014\\
San Diego, USA}
\def\theeventabbr{}
\def\theauthorabbr{M. Dénès \& M. Sozeau}
\usepackage{prelude-beamer}

\author{M. Dénès, \myname{}}
\def\theauthor{M. Dénès$^{\dagger}$ \& M. Sozeau$^{*}$}
\def\PI{\name{PI}}
\def\shaded#1{{\color{black!50}#1}}
 
\def\seq{\fCenter}

\def\lang{en}
\input{mathenv}
\setboolean{displayLabels}{true}

\pgfdeclareimage[height=1.4cm]{inria-logo}{inria-logo}

\pgfdeclareimage[height=1.4cm]{coq-biglogo}{barron-logo-pdf}
\pgfdeclareimage[height=0.4cm]{coq-smalllogo}{barron-logo}
\pgfdeclareimage[height=1cm,interpolate=true]{coq-logo}{barron-logo-pdf}

\pgfdeclareimage[height=7cm]{coq-share}{coq-share}

\def\imgheight{7.5cm}

\def\typea{\vdash}
\def\suba{\rightslice}
\def\indent{\coqdocindent}
\def\var{\coqdocvar}

\def\coqdocid#1{\coqdocvar{#1}}

%\renewcommand{\Type}[1]{\kw{Type}_{#1}}

\newcommand{\elt}[4]{\ensuremath{\constr{exist}_{#1,#2}~#3~#4}}
\newcommand{\eltlight}[2]{\ensuremath{\constr{exist}~#1~#2}}
\def\class{\kw{class}~}
\def\instance{\kw{instance}~}
\def\tclass#1{\coqdocind{#1}}
\def\where{\kw{where}~}
\def\module#1{\texttt{#1}}

\def\vector#1{\ind{vector}~#1}
\def\vnil{\constr{vnil}}
\def\vcons{\constr{vcons}}

\renewcommand{\bar}[1]{{\overline{#1}}}

\setlength{\coqdocbaseindent}{0em}

\renewcommand{\eqbr}{`=_{\beta}}

\def\coqlibrary#1#2#3{}

% Tutorial

\newboolean{darkfond}
\setboolean{darkfond}{false}
%\setboolean{darkfond}{true}

\definecolor{metavarcolor}{rgb}{0.5,0.0,1.0}
\definecolor{darkgreen}{rgb}{0.1,0.7,0.1}
\definecolor{answercolor}{rgb}{1.0,0.0,0.0}
\definecolor{normalcolor}{rgb}{0.0,0.0,0.0}
\definecolor{exbluecolor}{rgb}{0.1,0.1,0.9}
\definecolor{dontlookcolor}{rgb}{0.5,0.5,0.5}
\definecolor{termcolor}{rgb}{0.0,0.1,0.9}
\definecolor{lookcolor}{rgb}{0.8,0.2,0.0}
\definecolor{prooftermcolor}{rgb}{0.3,0.1,1.0}
\definecolor{typecolor}{rgb}{1.0,0.6,0.0}
\definecolor{taccolor}{rgb}{0.70,0.10,0.0}
\definecolor{pink}{rgb}{0.8,0.6,0.6}
\definecolor{darkmagenta}{rgb}{0.4,0.0,0.6}
\definecolor{darkblue}{rgb}{0.0,0.0,0.6}


\newcommand{\corkscrew}{\boldsymbol{\large \vdash}}
%\newcommand{\disj}{\mbox{$\backslash/$}}
%\newcommand{\conj}{\mbox{$/\backslash$}}
\newcommand{\juge}[3]{\mbox{$#1 \boldsymbol{\vdash} #2 : \color{red}#3 $}}
%\newcommand{\juge}[4]{\mbox{$#1,#2 \corkscrew #3 \boldsymbol{:} #4 $}}
%\newcommand{\smalljuge}[3]{\mbox{$#1 \corkscrew #2 \boldsymbol{:} #3 $}}
%\newcommand{\goal}[3]{\mbox{$#1,#2 \corkscrew^{\!\!\!?} #3  $}}
%\newcommand{\sgoal}[2]{\mbox{$#1\corkscrew^{\!\!\!\!?} #2 $}}
%\newcommand{\reduc}[5]{\mbox{$#1,#2 \corkscrew #3 \rhd_{#4}#5 $}}
%\newcommand{\convert}[5]{\mbox{$#1,#2 \corkscrew #3 =_{#4}#5 $}}
%\newcommand{\convorder}[5]{\mbox{$#1,#2 \corkscrew #3\leq _{#4}#5 $}}
%\newcommand{\wouff}[2]{\mbox{$\emph{WF}(#1)[#2]$}}


\newcommand{\mthese}{\underset{M}{\vdash}}
%\newcommand{\jthese}{\mathop{\vdash}_J}
%\newcommand{\jthese}{\stackrel[J]{}{\vdash}}
\newcommand{\jthese}{\underset{\tiny J}{\vdash}}
%\newcommand{\jthese}{\boldsymbol{\vdash}_{\tiny{ \!\!J}}}

\newcommand{\jsequent}[2]{\mbox{$#1\;\jthese\; #2 $}}
\newcommand{\msequent}[2]{\mbox{$#1\;\mthese\; #2 $}}



%\newcommand{\type}{\boldsymbol{:}}


\newcommand{\answ}{\color{answercolor}}
\newcommand{\coquser}{\color{normalcolor}}
\newcommand{\Ord}{\mbox{$\mathbb{O}$}}
\newcommand{\ords}{\mbox{\(\mathbb{O}\)}}
\newcommand{\nl}{\mbox{$\underset{\scriptstyle NL}{\longrightarrow}$}}
\newcommand{\base}[1]{{\color{lookcolor}#1}}
\newcommand{\om}{\mbox{$\omega$}}

%_{\alpha\beta\delta\iota}
\def\order{\mathcal{O}}
\def\tchecking#1#2#3{\ensuremath{#1 \vdash #2 : #3}}
\def\tconv#1#2#3#4{\ensuremath{#1 \vdash #2 = #3 "~>"
    #4}}
\def\tcumul#1#2#3#4{\ensuremath{#1 \vdash #2 \leq #3 "~>"
    #4}}

\def\tinfer#1#2#3#4#5#6{\ensuremath{#1; #2 \vdash #3 \Uparrow\ 
    "~>" #4 \vdash #5 : #6}}

\def\tcheck#1#2#3#4#5#6#7{\ensuremath{#1; #2 \vdash #3 \Downarrow #4
    "~>" #5 \vdash #6 : #7}}

\usepackage{mathpartir}
%\newcommand{\trule}[1]{\textsc{#1}}
%\newcommand{\irule}[3]
%{\inferrule{ #2 }{ #3 }{\ \trule{#1}}}
\newenvironment{subsecframe}[1]{\begin{subsection}{#1}\begin{frame}\frametitle{#1}}%
    {\end{frame}\end{subsection}}


\begin{document}


\setbeamertemplate{background canvas}[vertical shading][top=rouge1,middle=rouge1,bottom=rouge1]
\setbeamertemplate{footline}{\hspace{5em} \textcolor{white} {\null \hfill}\hspace{2em}\null \vspace*{3pt}}

\begin{frame}

\begin{textblock*}{40mm}[0,0](10mm,0mm)
 \includegraphics[width=6cm]{INRIA_CHERCHEURS_UK_RVB}
  \end{textblock*}

\begin{textblock*}{12cm}(13mm,50mm)
{\textcolor{white} {
{\huge \thetitle}\\[2mm]
{\theauthor}}}
\end{textblock*}


   \begin{textblock*}{40mm}[0,0](10mm,76mm)
  \begin{picture}(5,80)
\put(0,23){\includegraphics[width=4cm,height=1.5cm]{logobasrougeV1}}
\put(20,50){\tiny \textcolor{rouge2}{${}^{\dagger}$University of Pennsylvania}}
\put(20,40){\tiny \textcolor{rouge2}{${}^{*}$$πr^2$ - Inria \& PPS}}
\end{picture}
\end{textblock*}


\begin{textblock*}{7cm}(55mm,76mm)
{\textcolor{white}{{\theevent}}}
\end{textblock*}

\vspace*{-4pt}
\end{frame}


%%%%%%%%%%%%%%%%%%%%%%%%%%%

\setbeamertemplate{background canvas}[vertical shading][top=rouge1,bottom=rouge1,middle=rouge1]
\setbeamercolor{toto}{fg=blanc,bg=rouge1}

\setbeamertemplate{footline}
{
\begin{beamercolorbox}[wd=1\paperwidth,ht=15.5pt]{toto}
%\raisebox{1ex}{\includegraphics[width=25mm]{logobastrans}}
  \hspace{4em}
  \raisebox{2.5ex}
  {\thetitle{}  -- \theauthorabbr{}}\hfill 
  \raisebox{2.5ex}
  {\insertframenumber \hspace{2mm} \null }
\end{beamercolorbox}}

\setbeamertemplate{background canvas}[vertical shading][top=white,middle=white,bottom=white]

\def\bulletfail{\alert{\ding{54}}}
\def\bulletcheck{\ding{52}}

%\setbeamercolor{background}{fg=red,bg=white,text=blue}

%\setbeamercolor{subsection in toc shaded}{fg=gray,bg=white,text=blue}
\setbeamercolor{section in toc}{fg=black,bg=white,text=blue}
\setbeamercolor{subsection in toc}{fg=black,bg=white,text=blue}
%\setbeamertemplate{table of contents shaded}[default]
%\setbeamertemplate{subsection in toc shaded}{\textcolor{gray}}

\setbeamertemplate{section in toc shaded}
{\begin{colormixin}{20!white}{\usebeamertemplate{section in toc}}\end{colormixin}\unskip}

\setbeamertemplate{subsection in toc shaded}
{\begin{colormixin}{20!white}{\usebeamertemplate{subsection in toc}}\end{colormixin}\unskip}

\setbeamertemplate{subsubsection in toc shaded}
{\begin{colormixin}{20!white}{\usebeamertemplate{subsubsection in toc}}\end{colormixin}\unskip}


\frame<beamer>{\frametitle{\Coq 8.5}
\tableofcontents[sectionstyle=show/show,subsectionstyle=show/show/show]}

\section{Up and coming}
\begin{subsecframe}{Native compilation - M. Dénès}
\begin{itemize}
\item NbE style (?)
\item \Coq $"->"^{\texttt{nativecomp}}$ \textsc{ML} $"->"^{\texttt{ocamlopt}}$ \textsc{ASM}
\item Faster than \texttt{vm\_compute}
\end{itemize}

  \begin{itemize}
  \item[+] \emph{Optional}
  \item[++/--] \emph{Faster at runtime, compilation is slow}
  \end{itemize}
\end{subsecframe}

\begin{subsecframe}{Incremental development - E. Tassi}
  \begin{itemize}
  \item \texttt{git} model
  \item Granularity: opaque lemmas
  \item \texttt{.vi} interfaces: cross-file compilation
  \item CoqIDE only (emacs interface hard to implement)
  \end{itemize}

  \begin{itemize}
  \item[-/++] \emph{Not optional, backwards-compatible}
  \item[+++] \emph{Faster interaction and compilation}
  \end{itemize}
\end{subsecframe}

\begin{subsecframe}{Universe polymorphism - M. Sozeau}
% Def interactive
  \begin{itemize}
  \item \emph{Checked} universes.
  \item Global definitions or local with prenex, ``bounded''
    quantification. \emph{Conservativity proof}
  \item \emph{Elaboration} from typical ambiguity + implicit
    generalization.
  \item Checks \texttt{HoTT/Coq}, B. Barras and T. Coquand's semi-simplicial
    model and another (1-)groupoid model.
    \pause
  \item Better error messaging, unification sensitive to universes.
  \item All definitions and inductives can be polymorphic now.
  \end{itemize}

  \begin{itemize}
  \item[=/++] \emph{Optional except for checked
      universes, impacts the ML hacker only}. Backwards-compatibility layer.
  \item[=+/+] \emph{Comparable or better performance, more expressive}
  \end{itemize}
  
\end{subsecframe}

\begin{subsecframe}{Fast record projections - M. Sozeau}
  \begin{itemize}
  \item Economic representation
  \item Fast reduction
  \item No more stupid unfoldings
  \end{itemize}
  
  \begin{itemize}
  \item[+/+--] \emph{Optional, backwards-compatibility layer,
      small source-level incompatibilities}.
  \item[${+}^{ω}$] \emph{Exponentially better performance}
  \end{itemize}
  % * Graphe de perfs
\end{subsecframe}

\begin{subsecframe}{New proof engine - A. Spiwack}
  % * new internal proof engine (perfs)
  % - eauto backtrack with ; true proof search
  % "all:", tactical "+", "once", a native "refine"

  \begin{itemize}
  \item Existential variables + \texttt{LogicT}
  \item Proof-search semantics (\texttt{;}, \texttt{+}, $\ldots$)
  \item Cleaner code: abstract monadic interface
  \item Native \texttt{refine}, dependent goals, goal manipulation
  \end{itemize}
  
  \begin{itemize}
  \item[-/++] \emph{Not optional, backwards-compatibility layer}
  \item[--] \emph{0-15\% time overhead (still invesigating, \textsc{OCaml} issue likely)}
  \end{itemize}

\end{subsecframe}
\begin{subsecframe}{New coqdoc - Y. Régis-Gianas}
  \begin{itemize}
  \item Based on the AST and \alert{\emph{not}} lexing anymore
  \item All semantic information available
  \item Plugable interface
  \end{itemize}

  \begin{itemize}
  \item[+] \emph{Not optional but legacy \texttt{coqdoc} code included}
  \item[=] \emph{Same performance}
  \end{itemize}
  
  % \vfill
  % \begin{center}\alert{{\Huge Demo}}\end{center}
\end{subsecframe}
\end{document}

%%% Local Variables: 
%%% mode: latex
%%% TeX-PDF-mode: t
%%% TeX-master: "slides"
%%% End: 
